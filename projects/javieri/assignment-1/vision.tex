\documentclass[12pt]{article}
%\usepackage{biblatex}
\usepackage{cite}
\bibliography{myref}
%this is a comment
\title{Social Media Privacy 101}
\author{Megan Bigelow / bigelowm & Iliana Javier / javieri}




\begin{document}
\maketitle



\section{Project Goal}
To develop a web application providing accurate information and easy-to-follow instructions for maintaining privacy on the internet and social media sites.
\section{Why is Internet Privacy Important?}
Protecting our personal information is crucial as there are many ways that it can be used to hurt our identity and shape our behavior. Our personal data says a lot about ourselves as individuals and can be used to track our locations, habits, and interactions. This information can be used for fraudulent activity, to manipulate internet search results and advertisements, and allows our exact locations to be known by virtually anyone in the world. Having privacy on the internet allows us to feel confident that we can remain anonymous by keeping our personal data protected.
\section{Who is At Risk?}
Anyone with social media accounts or those that just use the internet for web surfing can be at risk of their personal data being compromised.

\section{Evidence That Internet Privacy is A Problem}
One of the biggest and most topical outcomes of compromised privacy is Doxing, which Dictionary.com defines as "To publish the private personal information of (another person) or reveal the identity of (an online poster) without the consent of that individual". According to NBC News's May 8 2017 video article "What is Doxing?", this breach in online privacy can lead to harassment in both a person's online and real lives, in some cases leading a person to losing their jobs. The general public is often not taking even the basic precautions to keep their privacy safe, highlighted by CNBC April 2014 article "Most Americans don't secure their smartphones" \cite{cnbc2014}, which showed only 7 percent of Americans in 2014 used any security features on their phones besides a screen lock. 


\section{How Will Our Solution Be More Useful Than Others?}
There are many articles and websites with information regarding internet privacy. The information is spread out, sporadic and does not always include instructions for applying privacy practices. Our web app will provide practical and up-to-date information regarding internet privacy and include easy-to-follow instructions for keeping personal data secure while using social media websites. The goal is to make the process of securing online privacy as fast and easy as possible.

\section{Limitations and Challenges}
Our web application would need to be updated regularly with the most reliable information available regarding internet privacy. We will need to research privacy policies and information to provide accurate content.

\section{References}
1. Weisbaum, Herb. "Most Americans don't secure their smartphones." CNBC, April 26, 2014. Accessed January 15, 2018. https://www.cnbc.com/2014/04/26/most-americans-dont-secure-their-smartphones.html \par 2. "Dox." Dictionary.com. Accessed January 15, 2018. http://www.dictionary.com/browse/dox




\bibliographystyle{plain}
\printbibliography

\end{document}